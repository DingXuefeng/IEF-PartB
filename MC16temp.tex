\documentclass[a4paper,11pt]{article}

\usepackage[T1]{fontenc}
\usepackage{lmodern}
\usepackage{eurosym}
\usepackage{lastpage}
\usepackage{xspace}
\usepackage[margin=16mm,includehead,includefoot]{geometry}
\usepackage{fancyhdr}
\usepackage{booktabs}
\usepackage{graphicx}
\usepackage{multirow}
\usepackage{array}
\usepackage{xcolor}
\usepackage{csquotes}
\usepackage{pgfgantt}
\usepackage{titlesec}
\usepackage[style=verbose-ibid,backend=bibtex]{biblatex}
\usepackage[colorlinks=true, citecolor=black, linkcolor=black, urlcolor=blue]{hyperref}
\usepackage{lscape}

\newcommand{\TODO}[1]{{\textcolor{red}{[\textbf{TODO:} #1]}}}
\newcommand{\acronym}{{\sc YourAcronym}\xspace}
\titlespacing\section{0pt}{6pt plus 4pt minus 2pt}{0pt plus 2pt minus 2pt}
\titlespacing\subsection{0pt}{6pt plus 4pt minus 2pt}{0pt plus 2pt minus 2pt}
\titlespacing\subsubsection{0pt}{6pt plus 4pt minus 2pt}{0pt plus 2pt minus 2pt}

\titleformat*{\section}{\large\bfseries}
\titleformat*{\subsection}{\normalsize\bfseries}
\titleformat*{\subsubsection}{\normalsize\bfseries}
\let\oldfootnotesize\footnotesize
\renewcommand{\footnotesize}{\fontsize{8bp}{1em}\selectfont}
\renewcommand{\cite}{\autocite} % citations in footnotes
\bibliography{bibliography}

\headheight=14pt

\pagestyle{fancy}
\fancyhead{}
\fancyhead[C]{\acronym \ -- GF}
\fancyfoot{}
\fancyfoot[C]{Part B - Page \thepage~of \pageref{LastPage}}


\renewcommand{\headrulewidth}{0pt}


\renewcommand{\contentsname}{TABLE OF CONTENTS}


\begin{document}

\phantom{a}
\vspace{15mm}
\begin{center}


        \Large{
      
     
        \textbf{Document 1\\\vspace{15mm}START PAGE}
  
          \vspace{15mm}
          MARIE SK\L{}ODOWSKA-CURIE ACTIONS\\
          \vspace{1cm}
          
          \textbf{Individual Fellowships (IF)}\\
          \textbf{Call: H2020-MSCA-IF-2016}
          \vspace{2cm}                   

          PART B
          \vspace{2.5cm}

          ``\acronym''
          \vspace{2cm}

          \textbf{This proposal is to be evaluated as:}
          \vspace{.5cm}

          \textbf{[GF]}
        }

  \end{center}
\vspace{1cm}

\newpage
%\setcounter{tocdepth}{1} %to remove subsections from table of contents
\tableofcontents


\newpage

\section*{LIST OF PARTICIPATING ORGANISATIONS}
\label{sec:participants}

Please provide a list of all participating organisations (both beneficiaries and, where applicable, partner organisations\footnote{All partner organisations should be listed here, including secondments}) indicating the legal entity, the department carrying out the work and the supervisor. 
If a secondment in Europe is planned but the partner organisation is not yet known, as a minimum the type of organisation foreseen (academic/non-academic) must be stated.
For non-academic beneficiaries, please provide additional data as indicated in the table below.

\newcommand\rotx[1]{\rotatebox[origin=c]{90}{\textbf{#1}}}
\newcommand\roty[1]{\rotatebox[origin=c]{90}{\parbox{4cm}{\raggedright\textbf{#1}}}}
\newcommand\MyHead[2]{\multicolumn{1}{l|}{\parbox{#1}{\centering #2}}}

\noindent\begin{tabular}{|m{2.4cm}|m{1cm}|b{1em}|b{1em}|c|m{2.5cm}|m{2cm}|c|}
\hline
  \textbf{Participants}
& \MyHead{1cm}{\textbf{Legal\\Entity\\Short\\Name}}
& \rotx{Academic}
& \rotx{Non-academic}
& \textbf{Country}
& \MyHead{2.1cm}{\textbf{Dept. / \\Division / \\Laboratory}}
& \textbf{Supervisor}
& \MyHead{2.5cm}{\textbf{Role of\\Partner\\Organisation\footnotemark}} \\
\hline
\underline{Beneficiary} & & & & & & & \\\hline
- NAME  & & & & & & & \\\hline
\underline{Partner} \underline{Organisation} & & & & & & & \\\hline
- NAME  & & & & & & & \\\hline
\end{tabular}
\vspace{\baselineskip}
\footnotetext{For example hosting secondments, for GF hosting the outgoing phase, etc.}

Data for non-academic beneficiaries

\noindent\begin{tabular}{|m{1.7cm}|m{2cm}|m{1.8cm}|c|c|m{2.5cm}|c|c|c|}
\hline
  \textbf{Name}
& \roty{Location of research premises (city / country)}
& \roty{Type of R\&D activities}
& \roty{No.~of full-time employees}
& \roty{No.~of employees in R\&D}
& \roty{Website}
& \roty{Annual turnover (approx. in Euro)}
& \roty{Enterprise status (Yes/No)}
& \roty{SME status\footnotemark  (Yes/No)}
\\\hline
& & & & & & & & \\\hline
\end{tabular}
\vspace{\baselineskip}
\footnotetext{As defined in \href{http://eur-lex.europa.eu/LexUriServ/LexUriServ.do?uri=OJ:L:2003:124:0036:0041:en:PDF}{Commission Recommendation 2003/361/EC.}}

Note that:
\begin{itemize}
\item Any inter-relationship between the participating organisation(s) or individuals and other entities/persons (e.g. family ties, shared premises or facilities, joint ownership, financial interest, overlapping staff or directors, etc.) must be declared and justified in this part of the proposal;
\item The information in the table for non-academic beneficiaries must be based on current data, not projections;
\end{itemize}


\newpage

\section{Excellence}
\label{sec:excellence}

\subsection{Quality and credibility of the research/innovation action (level of novelty, appropriate consideration of inter/multidisciplinary and gender aspects) }
\label{sec:quality}

You should develop your proposal according to the following lines:
\begin{itemize}
\item Introduction, state-of-the-art, objectives and overview of the action.
\item Research methodology and approach: highlight the type of research/innovation activities proposed.
\item Originality and innovative aspects of the research programme: explain the contribution that the action is expected to make to advancements within the action field. Describe any novel concepts, approaches or methods that will be employed.
\item The gender dimension in the research content (if relevant).
\item The interdisciplinary aspects of the action (if relevant).
\item Explain how the high-quality, novel research is the most likely to open up the best career possibilities for the experienced researcher and new collaboration opportunities for the host organisation(s).
\end{itemize}

\subsection{Quality and appropriateness of the training and of the two way transfer of knowledge between the researcher and the host}
\label{sec:transfer}

Describe the training that will be offered. 
Outline how a two way transfer of knowledge will occur between the researcher and the host institution(s):
\begin{itemize}
\item Explain how the experienced researcher will gain new knowledge during the fellowship at the hosting organisation(s)
\item Outline the previously acquired knowledge and skills that the researcher will transfer to the host organisation(s).
\end{itemize}

For Global Fellowships explain how the newly acquired skills and knowledge in the Third Country will be transferred back to the host institution in Europe (the beneficiary) during the incoming phase.

\subsection{Quality of the supervision and of the integration in the team/institution}
\label{sec:supervision}

\begin{itemize}
\item Qualifications and experience of the supervisor(s).
\end{itemize}

Provide information regarding the supervisor(s): the level of experience on the research topic proposed and their track record of work, including main international collaborations, as well as the level of experience in supervising researchers. Information provided should include participation in projects, publications, patents and any other relevant results.

\begin{itemize}
\item Hosting arrangements.\footnote{The hosting arrangements refer to the integration of the researcher to his new environment in the premises of the host. It does not refer to the infrastructure of the host as described in the Quality and efficiency of the implementation criterion.}
\end{itemize}

The application must show that the experienced researcher will be well integrated within the team/institution in order that all parties gain the maximum knowledge and skills from the fellowship. The nature and the quality of the research group/environment as a whole should be outlined, together with the measures taken to integrate the researcher in the different areas of expertise, disciplines, and international networking opportunities that the host could offer.

For GF both phases should be described - for the outgoing phase, specify the practical arrangements in place to host a researcher coming from another country, and for the incoming phase specify the measures planned for the successful (re-)integration of the researcher. 

\subsection{Capacity of the researcher to reach or re-enforce a position of professional maturity/independence}
\label{sec:maturity}

Applicants should demonstrate how the proposed research and training will contribute to the further professional development as an independent/mature researcher.

Describe \textbf{briefly} how the host will contribute to the advancement of the researcher's career.

Therefore, a complete \textbf{Career Development Plan should not be included in the proposal}, but it is part of implementing the action in line with the European Charter for Researchers.

\section{Impact}
\label{sec:impact}

\subsection{Enhancing the potential and future career prospects of the researcher }
\label{sec:enhancement}

Explain the expected impact of the planned research and training on the career prospects of the experienced researcher after the fellowship. Which new competences will be acquired? 

\subsection{Quality of the proposed measures to exploit and disseminate the action results}

Describe how the new knowledge generated by the action will be disseminated and exploited, e.g. communicated, transferred into other research settings or, if appropriate, commercialised. 

What is the dissemination strategy - targeted at scientists, potential users and to the wider research and innovation community - to achieve the potential impact of the action?

Please make also reference to the "Dissemination \& exploitation" section of the H2020 Online \href{http://ec.europa.eu/research/participants/docs/h2020-funding-guide/grants/grant-management/dissemination-of-results_en.htm}{Manual}.

The following section of the European Charter for Researchers refers specifically to dissemination:


\paragraph{Dissemination, exploitation of results}

All researchers should ensure, in compliance with their contractual arrangements, that the results of their research are disseminated and exploited, e.g. communicated, transferred into other research settings or, if appropriate, commercialised. Senior researchers, in particular, are expected to take a lead in ensuring that research is fruitful and that results are either exploited commercially or made accessible to the public (or both) whenever the opportunity arises.

Concrete planning for section 2.2 must be included in the Gantt Chart (see point 3.1).

\subsection{Quality of the proposed measures to communicate the action activities to different target audiences}

Please make also reference to the guidelines \href{http://ec.europa.eu/research/participants/data/ref/h2020/other/gm/h2020-guide-comm_en.pdf}{Communicating EU research and innovation guidance for project participants} as well as to the "communication" section of the H2020 Online \href{http://ec.europa.eu/research/participants/docs/h2020-funding-guide/grants/grant-management/communication_en.htm}{Manual}.
Concrete planning for section 2.3 must be included in the Gantt Chart (see point 3.1). 

The following section of the European Charter for Researchers refers specifically to public engagement:

\paragraph{Public engagement}

Researchers should ensure that their research activities are made known to society at large in such a way that they can be understood by non-specialists, thereby improving the public's understanding of science. Direct engagement with the public will help researchers to better understand public interest in priorities for science and technology and also the public's concerns. 

\section{Quality and Efficiency of the Implementation}
\label{sec:implementation}

\subsection{Coherence and effectiveness of the work plan}

The proposal should be designed in such a way to achieve the desired impact. A Gantt Chart should be included in the text listing the following:

\begin{itemize}
\item Work Packages titles (for EF there should be at least 1 WP); 
\item List of major deliverables, if applicable; \footnote{A deliverable is a distinct output of the action, meaningful in terms of the action?s overall objectives and may be a report, a document, a technical diagram, a software, etc. Should the applicants wish to participate in the pilot on Open Research Data, the Data Management Plan should be indicated here.

Deliverable numbers ordered according to delivery dates. Please use the numbering convention <WP number>.<number of deliverable within that WP>. For example, deliverable 4.2 would be the second deliverable from work package 4.}
\item List of major milestones, if applicable; \footnote{Milestones are control points in the action that help to chart progress. Milestones may correspond to the completion of a key deliverable, allowing the next phase of the work to begin. They may also be needed at intermediary points so that, if problems have arisen, corrective measures can be taken. A milestone may be a critical decision point in the action where, for example, the researcher must decide which of several technologies to adopt for further development.}
\item Secondments, if applicable.
\end{itemize}


The schedule should be in terms of number of months elapsed from the start of the action.


\paragraph{Example Gantt Chart}

Reflecting work package, secondments, training events and dissemination / public engagement activities
\begin{figure}[htbp]
\begin{center}

\begin{ganttchart}[
    x unit={0.45cm},
    canvas/.append style={fill=none, draw=black!5, line width=.75pt},
    hgrid style/.style={draw=black!5, line width=.75pt},
    vgrid={*1{draw=black!5, line width=.75pt}},
    title/.style={draw=none, fill=none},
    title label font=\bfseries\footnotesize,
    title label node/.append style={below=7pt},
    include title in canvas=false,
    bar label font=\small\color{black!70},
    group label font=\scriptsize,
    bar label node/.append style={left=2cm},
    bar/.append style={draw=none, fill=black!63},
    bar progress label font=\footnotesize\color{black!70},
    group left shift=0,
    group right shift=0,
    group height=.5,
    group peaks tip position=0,
    group label node/.append style={left=.1cm},
    group progress label font=\bfseries\small
  ]{1}{36}
  \gantttitle[
    title label node/.append style={below left=7pt and -6pt}
  ]{Month:\quad 1}{1}
  \gantttitlelist{2,...,36}{1} \\
  \ganttgroup[group label node/.append style={align=center}]{Work\ganttalignnewline Package}{1}{10} \\
  \ganttgroup{Deliverable}{5}{15} \\
  \ganttgroup{Milestone}{5}{5} \\
  \ganttgroup{Secondment}{20}{23} \\
  \ganttgroup{Conference}{16}{16} \\
  \ganttgroup{Workshop}{17}{17} \\
  \ganttgroup{Seminar}{18}{18} \\
  \ganttgroup{Dissemination}{23}{24} \\
  \ganttgroup[group label node/.append style={align=center}]{Public\ganttalignnewline engagement}{4}{5} \\
\ganttgroup{Other}{7}{10}
\end{ganttchart}

\end{center}
\end{figure}


\subsection{Appropriateness of the allocation of tasks and resources}

Describe how the work planning and the resources mobilised will ensure that the research and training objectives will be reached. 
Explain why the amount of person-months is appropriate in relation to the activities proposed.

\subsection{Appropriateness of the management structure and procedures, including risk management}

Describe the:

\begin{itemize}
\item Organisation and management structure, as well as the progress monitoring mechanisms put in place, to ensure that objectives are reached;
\item Research and/or administrative risks that might endanger reaching the action objectives and the contingency plans to be put in place should risk occur.
\end{itemize}


\subsection{Appropriateness of the institutional environment (infrastructure)}
\label{sec:institution}

The active contribution of the beneficiary to the research and training activities should be described. For GF also the role of partner organisations in Third Countries for the outgoing phase should appear.          
\begin{itemize}
\item Give a description of the main tasks and commitments of the beneficiary and all partner organisations (if applicable).
\item Describe the infrastructure, logistics, facilities offered in as far they are necessary for the good implementation of the action. 
\end{itemize}

%\newpage
%\vspace{15mm}
%\begin{center}
%
%
%        \Large{
%      
%     
%        \textbf{ENDPAGE}
%  
%          \vspace{15mm}
%          MARIE SK\L{}ODOWSKA-CURIE ACTIONS\\
%          \vspace{1cm}
%          
%          \textbf{Individual Fellowships (IF)}\\
%          \textbf{Call: H2020-MSCA-IF-2016}
%          \vspace{2cm}                   
%
%          PART B
%          \vspace{2.5cm}
%
%          ``\acronym''
%          \vspace{2cm}
%
%          \textbf{This proposal is to be evaluated as:}
%          \vspace{.5cm}
%
%          \textbf{[GF]}
%        }
%
%  \end{center}
%\vspace{1cm}

\end{document}